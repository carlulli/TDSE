\documentclass{article}

% Formatting
\usepackage[utf8]{inputenc}
\usepackage[margin=1in]{geometry}
\usepackage[titletoc,title]{}

% Math
% https://www.overleaf.com/learn/latex/Mathematical_expressions
% https://en.wikibooks.org/wiki/LaTeX/Mathematics
\usepackage{amsmath,amsfonts,amssymb,mathtools}

% Images
% https://www.overleaf.com/learn/latex/Inserting_Images
% https://en.wikibooks.org/wiki/LaTeX/Floats,_Figures_and_Captions
\usepackage{graphicx,float}   % needed for figures
\usepackage{subcaption}

% Tables
% https://www.overleaf.com/learn/latex/Tables
% https://en.wikibooks.org/wiki/LaTeX/Tables
\usepackage{tabularx}

% Algorithms
% https://www.overleaf.com/learn/latex/algorithms
% https://en.wikibooks.org/wiki/LaTeX/Algorithms
\usepackage[ruled,vlined]{algorithm2e}
\usepackage{algorithmic}

% Code syntax highlighting
% https://www.overleaf.com/learn/latex/Code_Highlighting_with_minted
\usepackage{minted}
\usemintedstyle{borland}

% References
% https://www.overleaf.com/learn/latex/Bibliography_management_in_LaTeX
% https://en.wikibooks.org/wiki/LaTeX/Bibliography_Management
\usepackage{biblatex}
\addbibresource{references.bib}

% to be able to click on links and references
\usepackage{hyperref}


% Title content
\title{Computational Physics II - Project 3 \\ Time-Dependet Schrödinger Equation}
\newcommand{\aone}{Adrian Stroth}
\newcommand{\atwo}{Edoardo Stefanin Mustacchi}
\newcommand{\athree}{Francisco Crespo}
\newcommand{\afour}{Cabibbo Group}
\author{\aone\space, \atwo\space and \athree \\ \afour}
\date{\today}

\begin{document}

\maketitle
\tableofcontents
\begin{abstract}
 
\end{abstract}

%  Code intallation & verification
\section{Code - TDSE}\label{code}

% subsection hamiltonian 
\subsection{From TISE}\label{TISE}

\subsection{Code Structure}\label{strct}

%
%subsection compiling and running 
\subsection{Compiling and Running of the Code}\label{cmplng}

% \begin{enumerate}
%     \item Run from within TISE \begin{verbatim}
%         main.o [N] [mass] [potential] [tolerance] [residue]
%     \end{verbatim}
%     \item Enter the desired parameters for simulation or many simulations in the {\fontfamily{qcr}\selectfont run\_simulation.sh} script and run from within TISE 
%         \begin{verbatim}
%             ./scripts/run_simulation.sh
%         \end{verbatim} 
%         or run the script from within scripts.
% \end{enumerate}

% section Algorithmic params
\section{Integrators}\label{integrator}

\subsection{Main choices}\label{mainchoices}

e.g. using kissfft etc.

\subsection{Testing}
% These can be put into subsubsections
Tests we implemented \\
Analytical, Linearity, Unitarity, Convergence
Testing strategy \\
How to run the test porgrams \\
Table with test parameters? \\
What will we know from the test once it is finished \\
show plots and dicuss why we think the tests were sucessful \\


% Example of table in latex
% \begin{center}
% \begin{tabular}{c|c|c|c|c}
%     Potential & N & $\hat{m}$ \\
%     \hline
%     Zero & 101 & 1\\
%     Harmonic & 1001 & 0.000005 \\
%     Well & 101 & 2.35 \\
%     Wall & 101 & 2.5 \\
% \end{tabular}
% \end{center}


\section{Investigation of Algorithmic Parameters}\label{algrtmcprmtrs}


% Example of how to include plots
% \begin{figure}
%     \centering
%     \includegraphics[scale = 0.7]{harmon_osc_continuum.png}
%     \caption{Varying the mass parameter m, and thereby varying a, while keeping the length L of the system constant. To achieve this the lattice point number N has to be adjusted according to $L = (N-1)*a$.}
%     \label{inf_volume_limit}
% \end{figure}


\section{Tunneling}


\clearpage
\section{Additional notes}
Github link to the code: \hyperlink{https://github.com/carlulli/TISE}{https://github.com/carlulli/TISE}.\\
Link to the files for future algorithmic parameter study:\\
\hyperlink{https://drive.google.com/drive/folders/1qDrw6Aoyzp01VN0UOLHqUtrzG8zt5tcV?usp=sharing}{https://drive.google.com/drive/folders/1qDrw6Aoyzp01VN0UOLHqUtrzG8zt5tcV?usp=sharing}

%\printbibliography
\end{document}
