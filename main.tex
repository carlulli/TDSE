\documentclass{article}

% Formatting
\usepackage[utf8]{inputenc}
\usepackage[margin=1in]{geometry}
\usepackage[titletoc,title]{}

% Math
% https://www.overleaf.com/learn/latex/Mathematical_expressions
% https://en.wikibooks.org/wiki/LaTeX/Mathematics
\usepackage{amsmath,amsfonts,amssymb,mathtools}

% Images
% https://www.overleaf.com/learn/latex/Inserting_Images
% https://en.wikibooks.org/wiki/LaTeX/Floats,_Figures_and_Captions
\usepackage{graphicx,float}   % needed for figures
\usepackage{subcaption}

% Tables
% https://www.overleaf.com/learn/latex/Tables
% https://en.wikibooks.org/wiki/LaTeX/Tables
\usepackage{tabularx}

% Algorithms
% https://www.overleaf.com/learn/latex/algorithms
% https://en.wikibooks.org/wiki/LaTeX/Algorithms
\usepackage[ruled,vlined]{algorithm2e}
\usepackage{algorithmic}

% Code syntax highlighting
% https://www.overleaf.com/learn/latex/Code_Highlighting_with_minted
\usepackage{minted}
\usemintedstyle{borland}

% References
% https://www.overleaf.com/learn/latex/Bibliography_management_in_LaTeX
% https://en.wikibooks.org/wiki/LaTeX/Bibliography_Management
\usepackage{biblatex}
\addbibresource{references.bib}

% to be able to click on links and references
\usepackage{hyperref}


% Title content
\title{Computational Physics II - Project 3 \\ Time-Dependet Schrödinger Equation}
\newcommand{\aone}{Adrian Stroth}
\newcommand{\atwo}{Edoardo Stefanin Mustacchi}
\newcommand{\athree}{Francisco Crespo}
\newcommand{\afour}{Cabibbo Group}
\author{\aone\space, \atwo\space and \athree \\ \afour}
\date{\today}

\begin{document}

\maketitle
\tableofcontents
\begin{abstract}
 
\end{abstract}

%  Code intallation & verification
\section{Code - TISE}\label{code}

% subsection hamiltonian 
\subsection{Hamiltonian}\label{hmltn}

\subsection{Main code}\label{main}

%subsection other modules 
\subsection{Side Modules}

%
%subsection compiling and running 
\subsection{Compiling and Running of the Code}
After downloading the code from \ref{code} on should have the necessary directory structure saved locally.
To compile all files on runs the script {\fontfamily{qcr}\selectfont compile.sh} from within TISE directory.
This creates a {\fontfamily{qcr}\selectfont main.o} inside TISE and executables of every test program inside the test directory.
Afterwards there exist two ways to run simulations. 
% \begin{enumerate}
%     \item Run from within TISE \begin{verbatim}
%         main.o [N] [mass] [potential] [tolerance] [residue]
%     \end{verbatim}
%     \item Enter the desired parameters for simulation or many simulations in the {\fontfamily{qcr}\selectfont run\_simulation.sh} script and run from within TISE 
%         \begin{verbatim}
%             ./scripts/run_simulation.sh
%         \end{verbatim} 
%         or run the script from within scripts.
% \end{enumerate}

% section Algorithmic params
\section{Study of the algorithmic parameters}\label{algrmtc parmtrs}



%What is the idea behind that?
%Why do we want to study the parameters?
%What are we expecting to see?\\

% \begin{center}
% \begin{tabular}{c|c|c|c|c}
%     Potential & N & $\hat{m}$ \\
%     \hline
%     Zero & 101 & 1\\
%     Harmonic & 1001 & 0.000005 \\
%     Well & 101 & 2.35 \\
%     Wall & 101 & 2.5 \\
% \end{tabular}
% \end{center}

\subsection{Zero-Potential}


\subsection{Effect of tolerance on power method}
The input parameters of the study can be taken from the table for each potential.
The tolerance was varied from 1 to $10^{-8}$. 

\subsection{Effect of residue on conjugate gradient}
The input parameters for the simulation are the same as above and can be seen in the table.
The residue parameters were varied from 1 to $10^{-10}$.

In both cases for high tolerance and residue values the observable values were varying from the cases with smaller tolerance or residue.

\section{Investigation of the Continuum Limit and the Infinite-Volume Limit}\label{finitevolume}
Compared to the continuum and infinite volume limits of the first project, the dependence is much simpler, when instead of a relativistic Schroedinger equation, the non-relativistic version is implemented, and furthermore there is only one dimension. 
To investigate the continuum limit the 
% Example of how to include plots
% \begin{figure}
%     \centering
%     \includegraphics[scale = 0.7]{harmon_osc_continuum.png}
%     \caption{Varying the mass parameter m, and thereby varying a, while keeping the length L of the system constant. To achieve this the lattice point number N has to be adjusted according to $L = (N-1)*a$.}
%     \label{inf_volume_limit}
% \end{figure}





\clearpage
\section{Additional notes}
Github link to the code: \hyperlink{https://github.com/carlulli/TISE}{https://github.com/carlulli/TISE}.\\
Link to the files for future algorithmic parameter study:\\
\hyperlink{https://drive.google.com/drive/folders/1qDrw6Aoyzp01VN0UOLHqUtrzG8zt5tcV?usp=sharing}{https://drive.google.com/drive/folders/1qDrw6Aoyzp01VN0UOLHqUtrzG8zt5tcV?usp=sharing}

%\printbibliography
\end{document}
